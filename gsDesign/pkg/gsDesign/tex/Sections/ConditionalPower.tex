\section{Conditional power and B-values\label{sec:CPB}}
\subsection{Group sequential test statistics as sums of independent increments\label{sec:increments}}
In some cases, rather than working with $Z_{1}$, $Z_{2}$,...,$Z_{k}$ as in Section \ref{sec:gsProbability}, it is desirable to consider
variables representing incremental sets of observations between analyses.
This approach will be applied here to define conditional power and B-values, two common measures of interim results and boundaries.
Letting $I_{0}=n_{0}=0$ we define $Y_{i}=\sum_{j=n_{i-1}+1}^{n_{i}}X_{j}%
/\sqrt{I_{i}-I_{i-1}}$ for $i=1,2,\ldots,k$. This implies $Y_{1}%
,Y_{2},...,Y_{k}$ are independent and normally distributed with
\begin{equation}
Y_{i}\sim N(\sqrt{I_{i}-I_{i-1}}\theta,1),\text{ }%
i=1,2,...,k.\label{Yi joint dist}%
\end{equation}
For $i=1,2,...,k$ if we let $w_{i}=\sqrt{I_{i}-I_{i-1}}$, note that
\begin{equation}
Z_{i}=\frac{\sum_{j=1}^{i}\sqrt{I_{j}-I_{j-1}}Y_{j}}{\sqrt{I_{i}}}=\frac
{\sum_{j=1}^{i}w_{j}Y_{j}}{\sqrt{\sum_{j=1}^{i}w_{j}^{2}}}%
.\label{Z sum of ind Y}%
\end{equation}
Finally, we define notation for independent increments between arbitrary
analyses. Select $i$ and $j$ with $1\leq i<j\leq k$ and let $Z_{i,j}%
=\sum_{m=n_{i}+1}^{n_{j}}X_{m}/\sqrt{I_{_{j}}-I_{i}}$. Thus, $Z_{i,j}\sim
N(\sqrt{I_{j}-I_{i}}\theta,1)$ is independent of $Z_{i}$\ and
\begin{equation}
Z_{j}=\frac{\sqrt{I_{i}}Z_{i}+\sqrt{I_{j}-I_{i}}Z_{i,j}}{\sqrt{I_{j}}%
}.\label{Zj as ind inc}
\end{equation}
By definition, for $i=2,3,...k$,
\begin{equation}
Y_{i}=Z_{i-1,i}.\label{Yi=Zi-1,i}%
\end{equation}
\bigskip

For the more general canonical form not defined using $X_{1},X_{2},...$ we
define $Y_{1}=Z_{1}$ and for $1\leq j<i\leq k$
\begin{equation}
Z_{j,i}=\frac{\sqrt{I_{i}}Z_{i}-\sqrt{I_{j}}Z_{j}}{\sqrt{I_{i}-I_{j}}%
}.\label{Zij implicit}%
\end{equation}
\bigskip The variables $Z_{j,i}$ and $Z_{j}$ are independent, as before, for
any $1\leq j<i\leq k$. We use (\ref{Yi=Zi-1,i}) to define $Y_{i}$,
$i=2,3,...,k.$ As before $Y_{i}\symbol{126}N(\sqrt{I_{i}-I_{i-1}}\theta,1)$,
$1<i\leq k,$ and these random variables are independent of each other.

\subsection{Conditional power\label{sec:CP}}

\bigskip
As an alternative to $\beta$-spending, stopping
rules for futility are interpreted by considering the conditional power of a
positive trial given the value of a test statistic at an interim analysis.
Thus, we consider the conditional probabities of boundary crossing for a group
sequential design given an interim result. Assume $1\leq$ $i<m\leq j\leq k$
and let $z_{i}$\ be any real value. Define%
\begin{equation}
u_{m,j}(z_{i})=\frac{u_{j}\sqrt{I_{j}}-z_{i}\sqrt{I_{m}}}{\sqrt{(I_{j}-I_{m}%
)}}\label{umj}%
\end{equation}
and
\begin{equation}
l_{m,j}(z_{i})=\frac{l_{j}\sqrt{I_{j}}-z_{i}\sqrt{I_{m}}}{\sqrt{(I_{j}-I_{m}%
)}}.\label{lmj}%
\end{equation}
Recall (\ref{Zj as ind inc}) and consider the conditional probabilities%
\begin{align}
\alpha_{i,j}(\theta|z_{i})  & =P_{\theta}\{\{Z_{j}\geqslant u_{j}%
\}\cap_{m=i+1}^{j-1}\{l_{m}<Z_{m}<u_{m}\}|Z_{i}=z_{i}%
\}\label{Cond lower bound prob}\\
& =P_{\theta}\left\{  \left\{  \frac{\sqrt{I_{i}}z_{i}+\sqrt{I_{j}-I_{i}%
}Z_{i,j}}{\sqrt{I_{j}}}\geqslant u_{j}\right\}  \cap_{m=i+1}^{j-1}\left\{
l_{m}<\frac{\sqrt{I_{i}}z_{i}+\sqrt{I_{j}-I_{i}}Z_{i,m}}{\sqrt{I_{j}}%
}\right\}  <u_{m}\right\} \nonumber\\
& =P_{\theta}\{\{Z_{i,j}\geqslant u_{i,j}(z_{i})\}\cap_{m=i+1}^{j-1}%
\{l_{m,j}(z_{i})<Z_{m,j}<u_{m,j}(z_{i})\}\}.\nonumber
\end{align}
This last line is of the same general form as $\alpha_{i}(\theta)$ and can
thus be computed in a similar fashion. For a non-binding bound, the same logic
applied ignoring the lower bound yields%

\begin{align}
\alpha_{i,j}^{+}(\theta|z_{i})  & =P_{\theta}\{\{Z_{j}\geqslant u_{j}%
\}\cap_{m=i+1}^{j-1}\{Z_{m}<u_{m}\}|Z_{i}=z_{i}\}\label{alphaij+}\\
& =P_{\theta}\{\{Z_{i,j}\geqslant u_{i,j}(z_{i})\}\cap_{m=i+1}^{j-1}%
\{Z_{m,j}<u_{m,j}(z_{i})\}\}.\nonumber
\end{align}
Finally, the conditional probability of crossing a lower bound at analysis $j$ given a test statistic $z_{i}$ at analysis $i$ is denoted by
\begin{align}
\beta_{i,j}(\theta|z_{i})  & =P_{\theta}\{\{Z_{j}\leq l_{j}\}\cap
_{m=i+1}^{j-1}\{l_{m}<Z_{m}<u_{m}\}|Z_{i}=z_{i}\}\label{Conditional power}\\
& =P_{\theta}\{\{Z_{i,j}\leq l_{i,j}(z_{i})\}\cap_{m=i+1}^{j-1}\{l_{m,j}
(z_{i})<Z_{m,j}<u_{m,j}(z_{i})\}\}.\nonumber
\end{align}
Since $\alpha_{i,j}^{+}(\theta|z_{i})$ and $\beta_{i,j}(\theta|z_{i})$\ are of
the same general form as $\alpha_{i}^{+}(\theta)$ and $\beta_{i}(\theta)$,
respectively, they can be computed using the same tools.

\subsection{B-values}
Proschan, Lan and Wittes \cite{PLWBook}.
