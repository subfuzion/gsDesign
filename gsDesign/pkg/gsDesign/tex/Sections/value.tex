\subsection{Calculating the value of a clinical trial design}
Here we generalize the concept of the probability of success of a trial given above to the value of a trial.
We assume that a trial that stops with a positive result with information $I_i$ at analysis $i$ of a trial when the true treatment effect is $\theta$ can be given by a function $u(\theta, I_i)$, $1\leq i\leq K$.
Now the formula for probability of success can be generalized to
\begin{equation}
U=\int_{-\infty}^\infty f(\theta) \sum_{i=1}^K \alpha_i(\theta)u(\theta,I_i)d\theta.\label{value}
\end{equation}
A more general formula that incorporates a costs that are incurred whether or not a trial is positive. 
If this formula also discounted future costs and benefits to present-day values, it would be termed a net present value and can be defined in a simplified form as
\begin{equation}
NPV=\int_{-\infty}^\infty f(\theta) \sum_{i=1}^K \alpha_i(\theta)u(\theta,I_i)-(\alpha_i(\theta)+\beta_i(\theta))
c(\theta,I_i)(d\theta.\label{NPV}
\end{equation}
While the underlying computations are not much more difficult to allow the utility and cost functions $u()$  and to depend on the test statistic at the time the trial stops, this capability has not been implemented in the package at this time. 
Arguably, however, the true value of a treatment depends on its true benefit rather than exactly what is observed in a clinical trial.

The following function computes the integral (\ref{value}).
\begin{verbatim}
value <- function(x, theta, ptheta, u)
{  x <- gsProbability(theta = theta, d=x)
   one <- array(1, x$k)
   as.real(one %*% (u * x$upper$prob) %*% ptheta)
}
\end{verbatim}
For this implementation, \code{u} must be a scalar, a vector of length \code{x\$k} or a matrix of the same dimension as \code{x\$upper\$prob} (\code{k} rows and \code{length(theta)} columns) rather than a function. 
We return to an example from the previous section.
Assuming $\theta\sim N(\mu=\delta, \sigma^2=(\delta/2)^2)$ we showed that the probability of success as defined in (\ref{POS}) is .748.
We will now change this definition so that a trial is considered a success only if it is positive and the true value of $\theta > \delta/2.$
This is computed as .521 as follows:
\begin{verbatim}
x <- gsDesign()
delta <- x$theta[2]
g <- normalGrid(mu=delta, sigma=delta / 2)
u <- 1 * (g$z > delta/2)
value(x, theta=g$z, ptheta=g$wgts, u=u)
\end{verbatim}

We finish with an example computing a futilty bound that optimizes the value of a design.
We will assume a fixed efficacy bound is used and will select an optimal spending function for the lower bound from a one-parameter family. 
We allow the user to specify the number of interim analyses as well as the desired Type I and Type II error and the prior distribution for the treatment effect.
We will assume a function \code{f(n, theta)} provides the value of a trial that stops for a positive result after enrolling \code{n} patients when the true treatment effect is \code{theta}. 
\begin{verbatim}
gsDesign(k=3, test.type=4, alpha=0.025, beta=0.1, astar=0,  
         delta=0, n.fix=1, timing=1, sfu=sfHSD, sfupar=-4,
         sfl=sfHSD, sflpar=-2, tol=0.000001, r=18, n.I = 0, maxn.IPlan = 0) 


function lbValue(sflpar=-2, k=3, test.type=4, alpha=0.025, beta=0.1, astar=0,  
         delta=0, n.fix=1, timing=1, sfu=sfHSD, sfupar=-4,
         sfl=sfHSD, tol=0.000001, r=18, f, theta, ptheta)
{   x <- gsDesign(
    u <- f()
    value(x, theta=theta, ptheta=ptheta, u=u)
}
\end{verbatim}
