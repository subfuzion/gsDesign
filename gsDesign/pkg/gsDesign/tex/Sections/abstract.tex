\title{gsDesign: An R Package for Designing \\Group Sequential Clinical Trials\\Version 2.0 Manual}
\author{Keaven M. Anderson\\Merck Research Laboratories}
\maketitle

\begin{abstract}
The gsDesign package supports group sequential clinical trial design. While
there is a strong focus on designs using $\alpha$- and $\beta$-spending
functions, Wang-Tsiatis designs, including O'Brien-Fleming and Pocock designs,
are also available. The ability to design with non-binding futility rules is
an important feature to control Type I error in a manner acceptable to
regulatory authorities.

The routines are designed to provide simple access to commonly used designs
using default arguments. Standard, published spending functions are supported
as well as the ability to write custom spending functions. A \texttt{gsDesign}
class is defined and returned by the \texttt{gsDesign()} function. A plot
function for this class provides a wide variety of plots: boundaries, power,
estimated treatment effect at boundaries, conditional power at boundaries,
spending function plots, expected sample size plot, and B-values at
boundaries. Using function calls to access the package routines provides a
powerful capability to derive designs or output formatting that could not be
anticipated through a gui interface. This enables the user to easily create
designs with features they desire, such as designs with minimum expected
sample size.

In addition to straightforward group sequential design, the gsDesign package provides tools to effectively adapt clinical trials during execution. 
First, the spending function approach to design allows altering timing of analyses during the course of the trial. 
Information-based timing of analyses allows adaptation of sample size or number of events to ensure adequate power for a trial.
Finally, gsDesign provides a routine that enable design adaptation using conditional power.

In summary, the intent of the gsDesign package is to easily create, fully
characterize, and even optimize routine group sequential trial designs, as well
as to provide a tool to derive and evaluate innovative designs.

\end{abstract}
