\subsection{Sequential $p$-values and confidence intervals}
\subsubsection{Boundary families, sample-space ordering and p-values}
Liu and Anderson \cite{AdaptExtend} have considered a family of cutoffs $b_i(\mu)$ 
which generalizes $b_i$ as previously defined, $i=1,2,\ldots,k.$
We assume for $i=1,2,\ldots,k$ that $b_i(\mu)$ is continuous and decreasing for $\mu\in(0,1)$ such that $b_i(\mu)\uparrow \infty$ as $\mu\downarrow 0$, $i=1,2,\ldots,k$
\begin{equation}
\mu=P_{0}\{\{Z_{i}\geq b_{i}(\mu)\}\cap_{j=1}^{i-1}%
\{Z_{j}<b_{j}(\mu)\}\}\label{eq:mu}.%
\end{equation}
Here we will also assume that $b_i(\mu)\downarrow -\infty$ as $\mu\uparrow 1$.


\subsection{Spending function families}
Suppose for some or all $\alpha\in (0,1)$ we defined a spending function $f(t;\alpha)$ where $f(1;\alpha)=\alpha$.
Now let $\xi>0$ as follows:
$$g(t;\xi)=f(t;\alpha)^{1/\xi}.$$ 
Note that for any $t$ for which $0<f(t)<1$, $g(t;\xi)$ approaches 0 as $\xi\downarrow 0$, approaches 1 as $\xi\uparrow \infty$ and equals $f(t;\alpha)$ when $\xi=1$. 
This can be reparameterized for an arbitrary $p\in (0,1)$ where $p$ is an alternate level of testing that is desired instead of $\alpha$.
A level $p$ spending function is obtained when $p^{1/\xi}=\alpha$ since $g(1;\xi)=p$ in this case. 
Writing $\xi$ in terms of $p$ we have $\xi=\ln p / \ln \alpha$.
We define the family of spending functions $f_p(t;\alpha)$ for $\alpha$ and $p$ in $(0, 1)$ by
\begin{equation}
f_p(t; \alpha)=f(t;\alpha)^{\ln p/\ln \alpha}
\end{equation}
As an example we let $f(t;\alpha, \rho)$ be the power spending function of Kim and DeMets \cite{KimDeMets}
$$f(t;\alpha,\rho)=\alpha\times t^\rho.$$
This implies that for $\rho>0$, $t\in [0,1]$ and $p$ and $\alpha$ in $(0,1)$
$$f_p(t;\alpha,\rho)=(\alpha\times t^\rho)^{\ln p/\ln(\alpha)}.$$

The exponential spending family fits nicely into this framework.
Recall that this family is defined for $t\in [0,1]$, $\alpha\in(0,1)$ and $\nu>0$ as
$$f(t;\alpha,\nu)=\alpha^{t^{-\nu}}.$$
Straightforward algebra shows that
$$f_p(t;\alpha)=p^{t^{-\nu}}=f(t;p).$$
