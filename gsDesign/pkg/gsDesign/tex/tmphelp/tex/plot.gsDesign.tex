\HeaderA{plot.gsDesign}{2.3: Plots for group sequential designs}{plot.gsDesign}
\aliasA{plot.gsProbability}{plot.gsDesign}{plot.gsProbability}
\aliasA{Plots for group sequential designs}{plot.gsDesign}{Plots for group sequential designs}
\keyword{design}{plot.gsDesign}
\begin{Description}\relax
The \code{plot()} function has been extended to work with objects returned by \code{gsDesign()} and \code{gsProbability()}. 
For objects of type \code{gsDesign}, seven types of plots are provided: z-values at boundaries (default), power, 
estimated treatment effects at boundaries, 
conditional power at boundaries, spending functions, expected sample size, and B-values at boundaries.
For objects of type \code{gsProbability} plots are available for z-values at boundaries, power (default), estimated treatment effects at boundaries, conditional power, expected sample size and B-values at boundaries.
\end{Description}
\begin{Usage}
\begin{verbatim}
plot.gsProbability(x, plottype=2, ...)
plot.gsDesign(x, plottype=1, ...)
\end{verbatim}
\end{Usage}
\begin{Arguments}
\begin{ldescription}
\item[\code{x}] Object of class \code{gsDesign} for \code{plot.gsDesign()} or \code{gsProbability} for 

\code{plot.gsProbability()}.
\item[\code{plottype}] 1=boundary plot (default for \code{gsDesign}),

2=power plot (default for \code{gsProbability}), 

3=estimated treatment effect at boundaries, 

4=conditional power at boundaries, 

5=spending function plot 
(only available if \code{class(x)=="gsDesign"}), 

6=expected sample size plot, and 

7=B-values at boundaries. 

Character values for \code{plottype} may also be entered: \code{"Z"} for plot
type 1,
\code{"power"} for plot type 2, \code{"thetahat"} for plot type 3,
\code{"CP"} for plot type 4, \code{"sf"} for plot type 5, \code{"ASN"}, 
\code{"N"} or \code{"n"} for plot type 6, and \code{"B"}, \code{"B-val"} or 
\code{"B-value"} for plot type 7.
\item[\code{...}] This allows many optional arguments that are standard when calling \code{plot}.

Other arguments include: 

\code{theta} which is used for \code{plottype=2}, \code{4}, \code{6}; normally defaults will be adequate; see details. 

\code{ses=TRUE} which applies only when \code{plottype=3} and 

\code{class(x)=="gsDesign"}; indicates that estimated standardized effect size
at the boundary is to be plotted rather than the actual estimate.

\code{xval="Default"} which is only effective when \code{plottype=2} or \code{6}. Appropriately scaled (reparameterized) values for x-axis for power and expected sample size graphs; see details.
\end{ldescription}
\end{Arguments}
\begin{Details}\relax
The intent is that many standard \code{plot()} parameters will function as expected; exceptions to this rule exist.
In particular, \code{main, xlab, ylab, lty, col, lwd, type, pch, cex} have been tested and work for most values of \code{plottype}; one
exception is that \code{type="l"} cannot be overridden when \code{plottype=2}. Default values for labels depend on \code{plottype} and 
the class of \code{x}.

Note that there is some special behavior for values plotted and returned 
for power and expected sample size (ASN) plots for a \code{gsDesign} object.
A call to \code{x<-gsDesign()} produces power and expected sample size for only two \code{theta} values: 0 and \code{x\$delta}. 
The call \code{plot(x, plottype="Power")} (or \code{plot(x,plottype="ASN"}) for a \code{gsDesign} object produces power (expected sample size) curves and returns a \code{gsDesign} object with \code{theta} values determined as follows. 
If \code{theta} is non-null on input, the input value(s) are used.
Otherwise, for a \code{gsProbability} object, the \code{theta} values from that object are used.
For a \code{gsDesign} object where \code{theta} is input as \code{NULL} (the default), \code{theta=seq(0,2,.05)*x\$delta}) is used. 
For a \code{gsDesign} object, the x-axis values are rescaled to \code{theta/x\$delta} and the label for the x-axis \eqn{theta / delta}{}.
For a \code{gsProbability} object, the values of \code{theta} are plotted and are labeled as \eqn{theta}{}.
See examples below.

Estimated treatment effects at boundaries are computed dividing the Z-values at the boundaries by the square root of \code{n.I} at that analysis. 

Spending functions are plotted for a continuous set of values from 0 to 1. 
This option should not be used if a boundary is used or a pointwise spending function is used
(\code{sfu} or \code{sfl="WT", "OF", "Pocock"} or \code{sfPoints}). 

Conditional power is computed using the function \code{gsBoundCP()}. 
The default input for this routine is \code{theta="thetahat"} which will compute the conditional power at each bound using the estimated treatment effect at that bound. 
Otherwise, if the input is \code{gsDesign} object conditional power is computed assuming \code{theta=x\$delta}, the original effect size for which the trial was planned.

Average sample number/expected sample size is computed using \code{n.I} at each analysis times the probability of crossing a boundary at that analysis.
If no boundary is crossed at any analysis, this is counted as stopping at the final analysis.

B-values are Z-values multiplied by \code{sqrt(t)=sqrt(x\$n.I/n\$n.I[x\$k])}. 
Thus, the expected value of a B-value at an analysis is the true value of 
\eqn{theta}{} multiplied by the proportion of total planned observations at that time.
See Proschan, Lan and Wittes (2006).
\end{Details}
\begin{Value}
An object of \code{class(x)}; in many cases this is the input value of \code{x}, while in others \code{x\$theta} is replaced 
and corresponding characteristics computed; see details.
\end{Value}
\begin{Note}\relax
The manual is not linked to this help file, but is available in library/gsdesign/doc/gsDesignManual.pdf
in the directory where R is installed.
\end{Note}
\begin{Author}\relax
Keaven Anderson \email{keaven\_anderson@merck.com}
\end{Author}
\begin{References}\relax
Jennison C and Turnbull BW (2000), \emph{Group Sequential Methods with Applications to Clinical Trials}.
Boca Raton: Chapman and Hall.

Proschan, MA, Lan, KKG, Wittes, JT (2006), \emph{Statistical Monitoring of Clinical Trials. A Unified Approach}. 
New York: Springer.
\end{References}
\begin{SeeAlso}\relax
\LinkA{gsDesign package overview}{gsDesign package overview}, \code{\LinkA{gsDesign}{gsDesign}}, \code{\LinkA{gsProbability}{gsProbability}}
\end{SeeAlso}
\begin{Examples}
\begin{ExampleCode}
#  symmetric, 2-sided design with O'Brien-Fleming-like boundaries
#  lower bound is non-binding (ignored in Type I error computation)
#  sample size is computed based on a fixed design requiring n=100
x <- gsDesign(k=5, test.type=2, n.fix=100)
x

# the following translate to calls to plot.gsDesign since x was
# returned by gsDesign; run these commands one at a time
plot(x)
plot(x, plottype=2)
plot(x, plottype=3)
plot(x, plottype=4)
plot(x, plottype=5)
plot(x, plottype=6)
plot(x, plottype=7)

#  choose different parameter values for power plot
#  start with design in x from above
y <- gsProbability(k=5, theta=seq(0, .5, .025), x$n.I,
                   x$lower$bound, x$upper$bound)

# the following translates to a call to plot.gsProbability since
# y has that type
plot(y)
\end{ExampleCode}
\end{Examples}

