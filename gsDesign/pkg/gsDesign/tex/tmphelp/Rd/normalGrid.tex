\HeaderA{normalGrid}{3.1: Normal Density Grid}{normalGrid}
\keyword{design}{normalGrid}
\begin{Description}\relax
normalGrid() is intended to be used for computation of the expected value of 
a function of a normal random variable.
The function produces grid points and weights to be used for numerical integration.
\end{Description}
\begin{Usage}
\begin{verbatim}
normalGrid(r=18, bounds=c(0,0), mu=0, sigma=1)
\end{verbatim}
\end{Usage}
\begin{Arguments}
\begin{ldescription}
\item[\code{r}] Control for grid points as in Jennison and Turnbull (2000), Chapter 19; default is 18. Range: 1 to 80.
This might be changed by the user (e.g., \code{r=6} which produces 65 gridpoints compare to 185 points when \code{r=18})
when speed is more important than precision.
\item[\code{bounds}] Range of integration. Real-valued vector of length 2. Default value of 0, 0 produces a range 
of + or - 6 standard deviations (6*sigma) from the mean (=mu).
\item[\code{mu}] Mean of the desired normal distribution.
\item[\code{sigma}] Standard deviation of the desired normal distribution.
\end{ldescription}
\end{Arguments}
\begin{Value}
\begin{ldescription}
\item[\code{z}] Grid points for numerical integration.
\item[\code{wgts}] Weights to be used with grid points in \code{z}.
\end{ldescription}
\end{Value}
\begin{Note}\relax
The manual is not linked to this help file, but is available in library/gsdesign/doc/gsDesignManual.pdf
in the directory where R is installed.
\end{Note}
\begin{Author}\relax
Keaven Anderson \email{keaven\_anderson@merck.com}
\end{Author}
\begin{References}\relax
Jennison C and Turnbull BW (2000), \emph{Group Sequential Methods with Applications to Clinical Trials}.
Boca Raton: Chapman and Hall.
\end{References}
\begin{Examples}
\begin{ExampleCode}
#  standard normal distribution
x <- normalGrid(r=3)
plot(x$z, x$wgts)

#  verify that numerical integration replicates sigma
#  get grid points and weights
x <- normalGrid(mu=2, sigma=3)

# compute squared deviation from mean for grid points
dev <- (x$z-2)^2

# multiply squared deviations by integration weights and sum 
sigma2 <- sum(dev * x$wgts)

# square root of sigma2 should be sigma (3)
sqrt(sigma2)

# do it again with larger r to increase accuracy
x <- normalGrid(r=22, mu=2, sigma=3)
sqrt(sum((x$z-2)^2 * x$wgts))

# find expected sample size for default design with
# n.fix=1000
x <- gsDesign(n.fix=1000)
x

y <- normalGrid(r=3, mu=x$theta[2], sigma=x$theta[2] / 1.5)
z <- gsProbability(k=3, theta=y$z, n.I=x$n.I, a=x$lower$bound,
                   b=x$upper$bound)
z <- gsProbability(d=x, theta=y$z)
cat("Expected sample size averaged over normal ")
cat("prior distribution for theta with mu=", 
   x$theta[2], "sigma=", x$theta[2]/1.5, ":",
   round(sum(z$en*y$wgt), 1), "\n")
plot(y$z, z$en, xlab="theta", ylab="E{N}", 
   main="Expected sample size for different theta values")
\end{ExampleCode}
\end{Examples}

