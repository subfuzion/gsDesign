\HeaderA{gsDesign package overview}{1.0 Group Sequential Design}{gsDesign package overview}
\keyword{design}{gsDesign package overview}
\begin{Description}\relax
gsDesign is a package for deriving and describing group sequential designs.
The package allows particular flexibility for designs with alpha- and beta-spending.
Many plots are available for describing design properties.
\end{Description}
\begin{Details}\relax
\Tabular{ll}{
Package: & gsDesign\\
Version: & 2\\
License: & GPL (version 2 or later)\\
}

Index:
\begin{alltt}
gsDesign                2.1: Design Derivation
gsProbability           2.2: Boundary Crossing Probabilities
plot.gsDesign           2.3: Plots for group sequential designs
gsCP                    2.4: Conditional Power Computation
gsBoundCP               2.5: Conditional Power at Interim Boundaries
gsbound                 2.6: Boundary derivation - low level
normalGrid              3.1: Normal Density Grid
binomial                3.2: Testing, Confidence Intervals and Sample Size
                             for Comparing Two Binomial Rates
Survival sample size    3.3: Time-to-event sample size calculation
                             (Lachin-Foulkes)
Spending function overview      4.0: Spending functions
sfHSD                   4.1: Hwang-Shih-DeCani Spending Function
sfPower                 4.2: Kim-DeMets (power) Spending Function
sfExponential           4.3: Exponential Spending Function
sfLDPocock              4.4: Lan-DeMets Spending function overview
sfPoints                4.5: Pointwise Spending Function
sfLogistic              4.6: 2-parameter Spending Function Families
sfTDist                 4.7: t-distribution Spending Function
Wang-Tsiatis Bounds     5.0: Wang-Tsiatis Bounds
checkScalar             6.0: Utility functions to verify variable properties
\end{alltt}
The gsDesign package supports group sequential clinical trial design. 
While there is a strong focus on designs using \eqn{\alpha}{alpha}- and \eqn{\beta}{beta}-spending functions, Wang-Tsiatis designs, 
including O'Brien-Fleming and Pocock designs, are also available.
The ability to design with non-binding futility rules 
allows control of Type I error in a manner acceptable to regulatory authorities when futility bounds are employed. 

The routines are designed to provide simple access to commonly used designs
using default arguments. 
Standard, published spending functions are supported as well as the ability to write custom spending functions. 
A \code{gsDesign} class is defined and returned by the \code{gsDesign()} function. 
A plot function for this class provides a wide variety of plots: boundaries, power, estimated treatment effect at boundaries, 
conditional power at boundaries, spending function plots, expected sample size plot, and B-values at boundaries.
Using function calls to access the package routines provides a powerful capability to derive designs or output 
formatting that could not be anticipated through a gui interface. 
This enables the user to easily create designs with features they desire, 
such as designs with minimum expected sample size.

Thus, the intent of the gsDesign package is to easily create, fully characterize and even 
optimize routine group sequential trial designs as well as provide a tool to evaluate innovative designs.
\end{Details}
\begin{Author}\relax
Keaven Anderson

Maintainer: Keaven Anderson \code{<keaven\_anderson@merck.com>}
\end{Author}
\begin{References}\relax
Jennison C and Turnbull BW (2000), \emph{Group Sequential Methods with Applications to Clinical Trials}.
Boca Raton: Chapman and Hall.

Proschan, MA, Lan, KKG, Wittes, JT (2006), \emph{Statistical Monitoring of Clinical Trials. A Unified Approach}. 
New York: Springer.
\end{References}
\begin{SeeAlso}\relax
\code{\LinkA{gsDesign}{gsDesign}}, \code{\LinkA{gsProbability}{gsProbability}}
\end{SeeAlso}
\begin{Examples}
\begin{ExampleCode}
# assume a fixed design (no interim) trial with the same endpoint
# requires 200 subjects for 90% power at alpha=.025, one-sided
x <- gsDesign(n.fix=200)
plot(x)
\end{ExampleCode}
\end{Examples}

